\subsection{Approach}
\label{subsec:approach}

Firstly, we will go through the theoretical basis and its possible implementation
in the context of our task, after which, we will see the algorithm in action.

\subsection{Theoretical basis}
\label{subsec:theoretical-basis}

The core of MinHashLSH algorithm is the utilization of two concepts: MinHash signatures and locality-sensitive hashing (LSH) to create and effective algorithm for detecting similar sets.
The first concept describes the process of hashing the dataset into a more manageable ``signature'' for each set.
While the latter describes how these ``signatures'' will be stored to achieve the expected result.

\subsection{Jaccard distance and Shingling}
\label{subsec:jaccard-distance-and-shingling}

Before performing any calculations to any data, we must convert the raw data into a distinguished vector before using any distance calculation between the two sets to check for their similarity.
Shingling performs this task by breaking down text data into smaller units to create ``shingles'' before hashing them into their representation in the form of a binary vector.

The Jaccard distance describes the similarity between two different sets and is represented by the following formula:
\begin{equation}
    \label{eq:equation}
    d_J(A, B) = \frac{|A \cup B| - |A \cap B|}{|A \cup B|} = 1 - J(A, B)
\end{equation}

With the result ranging from 0 to 1, we can determine whether the sets in question are related to each other or not for bucket assignment.

\subsection{MinHash Signature}
\label{subsec:minhash-signature}

To determine if the pairs are similar or not, we need a way to convert the raw data into usable data for the algorithm to process.
In this case,MinHashing is used to perform this task.
The binary vector, created through a process of shingling, is converted into a signature vector.
Note that if these sets 2 are similar, their signature vector will also have some similarities, and these can be utilized by the algorithm to sort these sets into their suitable buckets.

\subsection{Locality Sensitive Hashing}
\label{subsec:locality-sensitive-hashing}

The idea of LSH when dealing with the problem of finding similar pairs or sets is to maximize the probability of collision in a bucket due to that fact that the hash-code for these sets would be indifferent (if these sets are the similar) and therefore they should be in the same bucket.
We can achieve this by breaking the hash-code down even more into subsequences of hash-code that has a higher chance of being similar, giving us a higher chance of finding similar pairs, improving the effectiveness of the algorithm at solving the problem.