
\documentclass{article}
\usepackage{amsmath}
\usepackage{graphicx}

\title{PCY Algorithm for Frequent Itemset Mining and Association Rule Generation using PySpark}
\author{[Your Name], [Group Member 2], [Group Member 3], [Lecturer Name]}
\date{April 2025}

\begin{document}

\maketitle

\section{Introduction}
-[This introduction is for the whole of the project, delete this after transfer]-

In the age of big data, the ability to mine and extract valuable information from massive datasets can give the user an unparalleled edge against the competition. Therefore, this project is designed to simulate some of the three most fundamental challenges in data mining. Through these series of tasks, we will explore some algorithm implementations and solve different problems as well as explore their trade-offs 

The first task revolves around implementing the A-Priori algorithm in a Hadoop MapReduce program to discover groups of customers shopping in the same date as well as interacting with HDFS to store files and processing them. By applying these methods, we will be able to understand how to extract patterns from large datasets.

The second task focuses in implementing the PCY algorithm using OOP principles to discover patterns from the data. Compared to A-Priori, PCY provides some meaning and actionable metadata derived from the dataset that can be utilized later by the client.

The third task explores the MinHashLSH algorithm and an alternative manual implementation of jaccard distance to discover similar pairs of dates that pass a predetermined threshold with some visualization between both approach to understand their strengths and weaknesses.

Together, these tasks provides an outlines for some of the most basic operations on data mining techniques. This project not only provide some practical experience on these algorithms but also condition our mindset and form a general processing pipeline based on the characteristics of the given dataset. 


\begin{abstract}
-[This section is for the first task of the report - delete after transfer.]-
In this section of the report, we will implement a MapReduce version of the A-Priori algorithm to identify groups of customers that shop on the same dates. The program is designed to process the whole of the dataset and return the complete result after processing all data points.


\end{abstract}

\begin{abstract}
-[This section is for the second task of the report - delete after transfer.]-
For this section of the report, we will go through a different algorithm - the PCY algorithm while using PySpark DataFrame to identify frequent item pairs and generate association rules from customer purchase data stored in Google Drive. Our implementation includes a basic design of the hashing function and bucket management techniques while following object-oriented programming principles inspired by PySpark's FPGrowth class.



\end{abstract}

\begin{abstract}
-[This section is for the third task of the report - delete after transfer.]-
For our last section of the report, we will implement not only a proposed algorithm but also try and implement another way of our choosing, in our case, a manual Jaccard calculation method. Both of these approaches should achieve the some goal of searching for similar pairs of dates where the Jaccard distance is above a predetermined threshold. With all goals achieved, we will visualize their runtime from 0 to 1 with 0.1 increment to gauge their performance and outline some characteristics between both approaches.


\end{abstract}




\section{Approach}
Firstly, we will go through the theoretical basis and its possible implementation in the context of our task

\section{Theoretical basis}

\subsection{Underlying idea}


\subsection{}


\subsection{Second Pass: Counting Frequent Pairs}
During the second pass, the algorithm generates pairs 

\subsection{Association Rule Generation}
For each frequent pair, confidence is calculated 
\section{Code implementation}

\subsection{Underlying idea}


\subsection{}

\end{document}
