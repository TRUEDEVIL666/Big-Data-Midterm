\documentclass[conference]{IEEEtran}
\IEEEoverridecommandlockouts
% The preceding line is only needed to identify funding in the first footnote. If that is unneeded, please comment it out.
\usepackage{cite}
\usepackage{amsmath,amssymb,amsfonts}
\usepackage{algorithmic}
\usepackage{graphicx}
\usepackage{textcomp}
\usepackage{xcolor}
\usepackage{hyperref}
\usepackage{nohyperref}
\newcommand{\BibTeX}{\textrm{B \kern -.05em \textsc{i \kern -.025em b} \kern -.08em
T \kern -.1667em \lower .7ex \hbox{E} \kern -.125emX}}
\begin{document}

    \title{Mining Massive Data Sets Midterm Report}

    \author{
        \IEEEauthorblockN{1\textsuperscript{st} 522H0036 - Luong Canh Phong}
        \IEEEauthorblockA{
            \textit{Faculty of Information Technology} \\
            \textit{Ton Duc Thang University}\\
            Ho Chi Minh City, Vietnam \\
            522H0036@student.tdtu.edu.com
        }
        \and
        \IEEEauthorblockN{2\textsuperscript{nd} 522H0092 - Cao Nguyen Thai Thuan}
        \IEEEauthorblockA{
            \textit{Faculty of Information Technology} \\
            \textit{Ton Duc Thang University}\\ \
            Ho Chi Minh City, Vietnam \\
            522H0092@student.tdtu.edu.com
        }
        \and
        \IEEEauthorblockN{3\textsuperscript{rd} 522H0075 - Tang Minh Thien An}
        \IEEEauthorblockA{
            \textit{Faculty of Information Technology} \\
            \textit{Ton Duc Thang University}\\
            Ho Chi Minh City, Vietnam \\
            522H0075@student.tdtu.edu.com
        }
        \and
        \IEEEauthorblockN{4\textsuperscript{th} 522H0167 - Truong Tri Phong}
        \IEEEauthorblockA{
            \textit{Faculty of Information Technology} \\
            \textit{Ton Duc Thang University}\\
            Ho Chi Minh City, Vietnam \\
            522H0167@student.tdtu.edu.com
        }
        \and
        \IEEEauthorblockN{5\textsuperscript{th} Instructor: Nguyen Thanh An}
        \IEEEauthorblockA{
            \textit{Faculty of Information Technology} \\
            \textit{Ton Duc Thang University}\\
            Ho Chi Minh City, Vietnam \\
            nguyenthanhan@tdtu.edu.com
        }
    }

    \maketitle

    \begin{abstract}
        This document is a model and instructions for \LaTeX.
        This and the IEEEtran.cls file define the components of your paper [title, text, heads, etc.]. *CRITICAL: Do Not Use Symbols, Special Characters, Footnotes,
        or Math in Paper Title or Abstract.
    \end{abstract}

    \begin{IEEEkeywords}
        component, formatting, style, styling, insert
    \end{IEEEkeywords}

    \section{Introduction}
    \label{sec:introduction}

    I am not writing this part.

    \section{First Task: A-Priori Algorithm for Frequent Customers}
    \label{sec:first-task}
    \subsection{Overview of A-priori Algorithm}
\label{subsec:overview-of-a-priori-algorithm}

\subsubsection{What is the A-Priori Algorithm}

The A-Priori algorithm is a classic algorithm in data mining used to identify frequent itemsets and derive association rules.
In the context of this project, it is used to discover pairs of customers who frequently shop together.

\begin{itemize}
    \item It is based on the principle that all subsets of a frequent itemset must also be frequent.
    \item It uses a level-wise, breadth-first search approach to count the frequency of itemsets.
    \item It is typically implemented in multiple passes: the first pass finds frequent individual items (1-itemsets), and subsequent passes find larger frequent itemsets (e.g., 2-itemsets, 3-itemsets, etc.).
\end{itemize}

\subsubsection{How the A-Priori Algorithm Works (in this task)}

\begin{itemize}
    \item \textbf{First Pass – Frequent Individual Customers:} Count how many times each individual customer appears in the grouped transaction data.
    Customers with frequency above a defined threshold are considered frequent.
    \item \textbf{Second Pass – Frequent Customer Pairs:} For each transaction (i.e., group of customers on a given date), generate all possible customer pairs where both customers are frequent.
    Count the number of times each pair appears.
    \item \textbf{Support Threshold:} A predefined threshold used to filter out itemsets (customers or pairs) that do not appear frequently enough to be considered relevant.
    \item \textbf{Output:} The algorithm outputs customer pairs that occur together at or above the support threshold.
\end{itemize}

\subsection{First subtask}
\label{subsec:first-subtask}

In the first subtask, we are assigned to store the data on HDFS.
After which we will implement a Hadoop MapReduce program in Java to discover groups of customers going shopping at the same date.

\subsubsection{The Mapper Class} \textit{CustomerGroupByDateMapper}

The Mapper class is responsible for reading the input data and emitting key-value pairs.
Key aspects of its implementation include:

\begin{itemize}
    \item Input Processing: The input is a CSV file where each line contains multiple fields, including \texttt{Member\_number} (customer ID) and \texttt{Date} (transaction date).
    \item Filtering Headers: The Mapper skips header lines by checking if the first token equals \texttt{Member\_number}.
    \item Emitting Key-Value Pairs: The key is the transaction date, and the value is the customer ID.
    This allows transactions to be grouped by date in the shuffle and sort phase.
\end{itemize}

Example Output from Mapper:
\begin{center}
(01/01/2014, 11111)\\
(01/01/2014, 22222)\\
(02/01/2014, 11111)
\end{center}

\subsubsection{The Reducer Class} \textit{CustomerGroupByDateReducer}

The Reducer aggregates the values emitted by the Mapper for each unique date.
Key implementation features include:

\begin{itemize}
    \item Collecting Unique Customers: Customer IDs are added to a \texttt{HashSet} to remove duplicates.
    \item Joining Values: The set of unique customer IDs is converted to a comma-separated string.
    \item Emitting Results: The final output consists of the transaction date as the key, and a list of unique customer IDs as the value.
\end{itemize}

Example Output from Reducer:
\begin{center}
(01/01/2014, 11111,22222)\\
(02/01/2014, 11111)
\end{center}

\subsubsection{Driver Program (Main Method)}

The driver program configures and runs the MapReduce job.
It performs the following actions:

\begin{itemize}
    \item Job Setup: A Hadoop job is created with the name \texttt{Customer Date Groups}, using \texttt{GroupMapReduce} as the main class.
    \item Mapper and Reducer Assignment: The appropriate Mapper and Reducer classes are set.
    The Reducer is also used as a Combiner.
    \item I/O Paths: The input and output paths are taken from command-line arguments.
    \item Job Execution: The job is submitted to Hadoop, and the program exits based on success or failure.
\end{itemize}

\subsection{Second subtask}
\label{subsec:second-subtask}

In the second subtask, we implement the A-Priori algorithm to identify frequent customer pairs.
This is achieved using two MapReduce passes.

\textbf{Note:} The output of the first subtask (Customer Grouping by Date) will serve as the input for both passes of the second subtask (A-Priori Algorithm):

\begin{itemize}
    \item First Subtask (Customer Grouping by Date):
    \begin{itemize}
        \item \texttt{args[0]} – Input path to the raw transaction CSV file.
        \item \texttt{args[1]} – Output path where grouped customer data by date will be written.
    \end{itemize}

    \item Second Subtask (A-Priori Algorithm):
    \begin{itemize}
        \item \texttt{args[0]} – Input path to the grouped customer data (output from the first subtask).
        \item \texttt{args[1]} – Output path for the first pass (frequent individual customers).
        \item \texttt{args[2]} – Output path for the second pass (frequent customer pairs).
    \end{itemize}
\end{itemize}

This approach allows the workflow to seamlessly transition from the first subtask (grouping by date) to the second subtask (identifying frequent customers and pairs).

\subsubsection{The First Pass} Identifying Frequent Individual Customers

\begin{itemize}
    \item Mapper Class: \textit{AprioriFirstPassMapper}
    \begin{itemize}
        \item Function: Reads grouped customer data (output of first subtask), splits the customer list, and emits each customer ID with a value of 1.
        \item Input Format: Each line is a tab-separated pair where the key is a date and the value is a comma-separated list of customers.
        \item Filtering: Skips malformed lines where the customer list is missing.
        \item Example Input:
        \begin{center}
            01/01/2014 \tab 12345,67890
        \end{center}
        \item Example Output from Mapper:
        \begin{center}
        (12345, 1)\\
        (67890, 1)
        \end{center}
    \end{itemize}

    \item Reducer Class: \textit{AprioriReducer}
    \begin{itemize}
        \item Function: Aggregates the counts for each key (customer).
        \item Filtering: Emits only \texttt{<key, value>} pairs whose frequency is greater than or equal to the support threshold.
        \item Example Output from Reducer (First Pass):
        \begin{center}
        (12345, 2)\\
        (67890, 1)
        \end{center}
    \end{itemize}
\end{itemize}

\subsubsection{The Second Pass} \textit{Identifying Frequent Customer Pairs}

\begin{itemize}
    \item Mapper Class: \textit{AprioriSecondPassMapper}
    \begin{itemize}
        \item Setup: Loads the list of frequent customers from the first pass output using Hadoop’s distributed cache.
        \item Processing: For each transaction line, splits the list of customer IDs, filters only frequent customers, and generates all valid customer pairs.
        \item Emitting: Outputs each pair of frequent customers with a count of 1.
        \item Example Output from Mapper:
        \begin{center}
        (12345,67890, 1)\\
        (12345,54321, 1)
        \end{center}
    \end{itemize}

    \item Reducer Class: \textit{AprioriReducer}
    \begin{itemize}
        \item The same class as used in the First Pass.
        \item Example Output from Reducer (Second Pass):
        \begin{center}
        (12345,67890, 3)
        \end{center}
    \end{itemize}
\end{itemize}

\subsubsection{Driver Program (Main Method)}

\begin{itemize}
    \item First Pass Execution:
    \begin{itemize}
        \item The first MapReduce job is run with \texttt{AprioriFirstPassMapper} to compute individual customer frequencies.
        \item The output is saved and later loaded into memory for the second pass.
    \end{itemize}

    \item Second Pass Execution:
    \begin{itemize}
        \item The second job uses \texttt{AprioriSecondPassMapper}, which loads the frequent customers using Hadoop's cache mechanism.
        \item It then computes the frequency of customer pairs and applies the support threshold.
    \end{itemize}
\end{itemize}

\subsubsection{Helper Method} \textit{createJob}

A reusable helper method named \texttt{createJob} is implemented to reduce code repetition when setting up MapReduce jobs.

\begin{itemize}
    \item Parameters:
    \begin{itemize}
        \item \texttt{jobName}: A string representing the name of the job.
        \item \texttt{mapperClass}: The class to be used as the Mapper.
        \item \texttt{inputPath}, \texttt{outputPath}: Paths for input and output directories.
    \end{itemize}
    \item Functionality:
    \begin{itemize}
        \item Configures the job with the specified name and sets the \texttt{AprioriReducer} as both the Combiner and Reducer.
        \item Assigns key and value output types and adds file paths.
        \item Returns a configured \texttt{Job} instance ready for execution.
    \end{itemize}
\end{itemize}

    \section{Second Task: PCY Algorithm for Frequent Items}
    \label{sec:second-task}
    \subsection{Overview of PCY}
\label{subsec:overview-of-pcy}
Frequent itemsets mining is essential for discovering item associations in transactional data, such as market basket analysis.
The PCY algorithm improves efficiency by using hash buckets to reduce the computational cost of finding frequent item pairs.
This project applies the PCY algorithm to mine frequent itemsets and generate association rules based on support and confidence thresholds, using PySpark for scalable data processing.

The PCY algorithm is based on two key passes through the data.
In the first pass, frequent individual items are identified and counted.
In the second pass, frequent item pairs are counted, and hash buckets are used to prune less frequent pairs.
The hash function maps item pairs to buckets, and only pairs that have a sufficient bucket count are considered frequent.
This approach significantly reduces the number of pair comparisons and improves algorithm efficiency.

\begin{itemize}
    \item Step 1: Count individual items using the support threshold.
    \item Step 2: Count pairs of frequent items and hash them into buckets.
    \item Step 3: Prune item pairs that are not frequent based on the bucket counts.
    \item Step 4: Generate association rules using the confidence threshold.
\end{itemize}

\subsection{Implementation Details}
\label{subsec:implementation-details}

\subsubsection{Data Loading and Preprocessing}
Data is loaded using PySpark’s \texttt{read.csv} function.
Each transaction is represented as a basket, and the data is grouped by customer and date.
The \texttt{collect\_set} function is used to create a list of items bought together in each transaction.

\subsubsection{First Pass: Counting Frequent Items}
In the first pass, each item’s frequency is counted, and only those items that meet the support threshold are considered frequent.
The item counts are stored in a dictionary, sorted in descending order.

\subsubsection{Second Pass: Counting Frequent Pairs}
During the second pass, the algorithm generates pairs from frequent items and counts their occurrences.
Hashing is applied to map pairs into buckets, and the bucket counts are used to prune pairs that do not meet the minimum support threshold.

\subsubsection{Association Rule Generation}
For each frequent pair, confidence is calculated as the ratio of the pair’s count to the individual item count.
Association rules are generated if the confidence meets the given threshold.
The rules are sorted by confidence.

\subsection{Experimental Results}
\label{subsec:experimental-results}
The PCY algorithm was executed on a transactional dataset of retail transactions, where each transaction (basket) represented a set of items purchased by a customer.
The algorithm was applied with the following parameters:
\begin{itemize}
    \item Support threshold: 2 (minimum count for items to be considered frequent).
    \item Confidence threshold: 0.5 (minimum confidence for association rules).
\end{itemize}

\subsubsection{Frequent Items}
The first pass of the algorithm counted the occurrence of each item in the transactions.
The following are the top 30 frequent items identified based on the support threshold:

\begin{table}[h]
    \centering
    \caption{Most Frequent Items with Their Counts}
    \setlength{\tabcolsep}{2pt} % Reduce column spacing
    \renewcommand{\arraystretch}{1} % Adjust row spacing
    \resizebox{120}{!}{ % Fit within column width
        \begin{tabular}{|l|c|}
            \hline
            \textbf{Item} & \textbf{Count} \\
            \hline
            Whole milk & 2363 \\
            Other vegetables & 1827 \\
            Rolls/buns & 1646 \\
            Soda & 1453 \\
            Yogurt & 1285 \\
            Root vegetables & 1041 \\
            Tropical fruit & 1014 \\
            Bottled water & 908 \\
            Sausage & 903 \\
            Citrus fruit & 795 \\
            Pastry & 774 \\
            Pip fruit & 734 \\
            \hline
        \end{tabular}
    }
    \label{tab:frequent_items}
\end{table}

\subsubsection{Frequent Item Pairs}
In the second pass, the algorithm counted pairs of frequent items across all transactions.
The top 30 frequent item pairs, sorted by frequency, are as follows:

\begin{table}[h]
    \centering
    \caption{Frequent Item Pairs with Their Counts}
    \setlength{\tabcolsep}{2pt} % Reduce column spacing
    \renewcommand{\arraystretch}{1} % Adjust row spacing
    \resizebox{160}{!}{ % Fit within column width
        \begin{tabular}{|l|c|}
            \hline
            \textbf{Pair} & \textbf{Count} \\
            \hline
            ('Whole milk', 'Other vegetables') & 222 \\
            ('Whole milk', 'Rolls/buns') & 209 \\
            ('Whole milk', 'Soda') & 174 \\
            ('Whole milk', 'Yogurt') & 167 \\
            ('Rolls/buns', 'Other vegetables') & 158 \\
            ('Soda', 'Other vegetables') & 145 \\
            ('Whole milk', 'Sausage') & 134 \\
            ('Whole milk', 'Tropical fruit') & 123 \\
            ('Yogurt', 'Other vegetables') & 121 \\
            ('Rolls/buns', 'Soda') & 121 \\
            ('Yogurt', 'Rolls/buns') & 117 \\
            ('Whole milk', 'Root vegetables') & 113 \\
            \hline
        \end{tabular}
    }
    \label{tab:frequent_pairs}
\end{table}

\subsubsection{Association Rules}
Once frequent item pairs were identified, association rules were generated based on the confidence threshold of 0.5. The confidence for each rule was computed by dividing the pair count by the count of the antecedent item.
The top 30 association rules, sorted by confidence, are presented below:

\begin{table}[h]
    \centering
    \caption{Validated Association Rules}
    \setlength{\tabcolsep}{2pt} % Reduce column spacing
    \renewcommand{\arraystretch}{1} % Adjust row spacing
    \resizebox{180}{!}{ % Fit within column width
        \begin{tabular}{|l|c|}
            \hline
            \textbf{Rule} & \textbf{Confidence} \\
            \hline
            Preservation products → Soups & 1.00 \\
            Kitchen utensil → Pasta & 1.00 \\
            Kitchen utensil → Bottled water & 1.00 \\
            Kitchen utensil → Rolls/buns & 1.00 \\
            Bags → Yogurt & 0.50 \\
            \hline
        \end{tabular}
    }
    \label{tab:association-rules}
\end{table}

\subsubsection{Performance Evaluation}
The PCY algorithm effectively identified frequent items, pairs, and association rules with reasonable execution time and memory usage.
Given the dataset size, the distributed nature of Spark ensured that the computation was scalable.
\begin{itemize}
    \item Time Complexity: The use of hashing significantly reduces the complexity of item pair generation, making the PCY algorithm faster than traditional algorithms such as the Apriori algorithm.
    \item Memory Usage: Memory usage was managed well by leveraging the distributed processing capabilities of Spark.
\end{itemize}

    \section{Ease of Use}
    \label{sec:ease-of-use}

    \subsection{Maintaining the Integrity of the Specifications}
    \label{subsec:maintaining-the-integrity-of-the-specifications}

    The IEEEtran class file is used to format your paper and style the text.
    All margins, column widths, line spaces, and text fonts are prescribed; please do not alter them.
    You may note peculiarities.
    For example, the head margin measures proportionately more than is customary.
    This measurement and others are deliberate, using specifications that anticipate your paper as one part of the entire proceedings, and not as an independent document.
    Please do not revise any of the current designations.

    \section{Prepare Your Paper Before Styling}
    \label{sec:prepare-your-paper-before-styling}
%    \input{sections/prepare-your-paper-before-styling}

    \section*{Acknowledgment}

    The preferred spelling of the word ``acknowledgment'' in America is without an ``e'' after the ``g''.
    Avoid the stilted expression ``one of us (R. B. G.) thanks $\ldots$''.
    Instead, try ``R. B. G. thanks$\ldots$''.
    Put sponsor acknowledgments in the unnumbered footnote on the first page.

    \section*{References}

    Please number citations consecutively within brackets~\cite{b1}.
    The sentence punctuation follows the bracket~\cite{b2}.
    Refer simply to the reference number, as in~\cite{b3}---do not use ``Ref. \cite{b3}'' or ``reference~\cite{b3}'' except at the beginning of a sentence: ``Reference~\cite{b3} was the first $\ldots$''

    Number footnotes separately in superscripts.
    Place the actual footnote at
    the bottom of the column in which it was cited.
    Do not put footnotes in the abstract or reference list.
    Use letters for table footnotes.

    Unless there are six authors or more give all authors' names; do not use ``et al.''.
    Papers that have not been published, even if they have been submitted for publication, should be cited as ``unpublished''~\cite{b4}.
    Papers that have been accepted for publication should be cited as ``in press''~\cite{b5}.
    Capitalize only the first word in a paper title, except for proper nouns and element symbols.

    For papers published in translation journals, please give the English
    citation first, followed by the original foreign-language citation~\cite{b6}.

    \begin{thebibliography}{00}
        \bibitem{b1} G. Eason, B. Noble, and I. N. Sneddon, ``On certain integrals of Lipschitz-Hankel type involving products of Bessel functions,'' Phil. Trans. Roy. Soc. London, vol. A247, pp. 529--551, April 1955.
        \bibitem{b2} J. Clerk Maxwell, A Treatise on Electricity and Magnetism, 3rd ed., vol. 2. Oxford: Clarendon, 1892, pp.68--73.
        \bibitem{b3} I. S. Jacobs and C. P. Bean, ``Fine particles, thin films and exchange anisotropy,'' in Magnetism, vol. III, G. T. Rado and H. Suhl, Eds. New York: Academic, 1963, pp. 271--350.
        \bibitem{b4} K. Elissa, ``Title of paper if known,'' unpublished.
        \bibitem{b5} R. Nicole, ``Title of paper with only first word capitalized,'' J. Name Stand. Abbrev., in press.
        \bibitem{b6} Y. Yorozu, M. Hirano, K. Oka, and Y. Tagawa, ``Electron spectroscopy studies on magneto-optical media and plastic substrate interface,'' IEEE Transl. J. Magn. Japan, vol. 2, pp. 740--741, August 1987 [Digests 9th Annual Conf. Magnetics Japan, p. 301, 1982].
        \bibitem{b7} M. Young, The Technical Writer's Handbook. Mill Valley, CA: University Science, 1989.
    \end{thebibliography}
    \vspace{12pt}
    \color{red}
    IEEE conference templates contain guidance text for composing and formatting conference papers. Please ensure that all template text is removed from your conference paper prior to submission to the conference. Failure to remove the template text from your paper may result in your paper not being published.

\end{document}