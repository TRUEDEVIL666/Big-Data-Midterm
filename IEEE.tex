\documentclass[conference]{IEEEtran}
\IEEEoverridecommandlockouts
% The preceding line is only needed to identify funding in the first footnote. If that is unneeded, please comment it out.
\usepackage{cite}
\usepackage{amsmath,amssymb,amsfonts}
\usepackage{algorithmic}
\usepackage{graphicx}
\usepackage{textcomp}
\usepackage{xcolor}
\usepackage{hyperref}
\usepackage{nohyperref}
\newcommand{\BibTeX}{\textrm{B \kern -.05em \textsc{i \kern -.025em b} \kern -.08em
T \kern -.1667em \lower .7ex \hbox{E} \kern -.125emX}}
\begin{document}

    \title{Mining Massive Data Sets Midterm Report}

    \author{
        \IEEEauthorblockN{1\textsuperscript{st} 522H0036 - Luong Canh Phong}
        \IEEEauthorblockA{
            \textit{Faculty of Information Technology} \\
            \textit{Ton Duc Thang University}\\
            Ho Chi Minh City, Vietnam \\
            522H0036@student.tdtu.edu.com
        }
        \and
        \IEEEauthorblockN{2\textsuperscript{nd} 522H0092 - Cao Nguyen Thai Thuan}
        \IEEEauthorblockA{
            \textit{Faculty of Information Technology} \\
            \textit{Ton Duc Thang University}\\ \
            Ho Chi Minh City, Vietnam \\
            522H0092@student.tdtu.edu.com
        }
        \and
        \IEEEauthorblockN{3\textsuperscript{rd} 522H0075 - Tang Minh Thien An}
        \IEEEauthorblockA{
            \textit{Faculty of Information Technology} \\
            \textit{Ton Duc Thang University}\\
            Ho Chi Minh City, Vietnam \\
            522H0075@student.tdtu.edu.com
        }
        \and
        \IEEEauthorblockN{4\textsuperscript{th} 522H0167 - Truong Tri Phong}
        \IEEEauthorblockA{
            \textit{Faculty of Information Technology} \\
            \textit{Ton Duc Thang University}\\
            Ho Chi Minh City, Vietnam \\
            522H0167@student.tdtu.edu.com
        }
        \and
        \IEEEauthorblockN{5\textsuperscript{th} Instructor: Nguyen Thanh An}
        \IEEEauthorblockA{
            \textit{Faculty of Information Technology} \\
            \textit{Ton Duc Thang University}\\
            Ho Chi Minh City, Vietnam \\
            nguyenthanhan@tdtu.edu.com
        }
    }

    \maketitle

    \begin{abstract}
        In the age of big data, the ability to mine and extract valuable information from massive datasets can give the user an unparalleled edge against the competition.
        Therefore, this requirement made by the lecturer is designed to simulate one of the three most fundamental challenges in data mining.
        Through these series of tasks, we will explore some algorithm implementations and solve different problems as well as explore their trade-offs.
        Each task is a different algorithm to explore and implement with their corresponding datasets.
        Through these tasks, we will gain some practical insight and experience in working with these algorithms as well as a better understanding of their pros and cons to be able to cater to each dataset based on their characteristics.
    \end{abstract}

%    \begin{IEEEkeywords}
%        component, formatting, style, styling, insert
%    \end{IEEEkeywords}

    \section{Introduction}
    \label{sec:introduction}

    This report is divided into three large sections corresponding to the first three tasks provided by the lecturer.
    We will explore and present our findings while putting the proposed algorithms into practice.

    Task 1 proposes utilizing the A-Priori algorithm in a Hadoop MapReduce program to discover groups of customers shopping on the same date as well as interacting with Hadoop Distributes File System (HDFS) to store files.
    By applying these methods, we will be able to understand how to extract patterns from large datasets locally.

    The second task focuses on implementing the Park-Chen-Yu (PCY) algorithm using Object-Oriented Programming (OOP) principles and PySpark DataFrame to identify frequent item pairs and generate association rules from customer purchase data stored in Google Drive.
    The implementation, while generating association rules, also has to follow object-oriented programming principles inspired by PySpark's Frequent-Pattern Growth (FPGrowth) class.

    In the third task, we will implement and compare the MinHashLSH algorithm and an alternative of our choice - in this case, a manual method of calculating Jaccard distance.
    Both of these approaches should achieve the same goal of searching for similar pairs of dates where the Jaccard distance is above a predetermined threshold.
    After that, we will visualize their runtime with their threshold ranging from 0 to 1 with 0.1 increments to gauge their performance and outline some characteristics between both approaches.
    Through these implementations, we demonstrate practical applications of data mining techniques with a given dataset.
    With these findings, we highlight the trade-offs between various aspects across different algorithms within the given time and constraints.

    \section{First Task: A-Priori Algorithm for Frequent Customers}
    \label{sec:first-task}
    \subsection{Overview of MapReduce}
\label{subsec:overview-of-mapreduce}

\subsubsection{What is MapReduce}

MapReduce is a yarn-based system commonly used for processing massive dataset:

\begin{itemize}
    \item Performs concurrent processing by dividing the dataset into multiple chunks on the Hadoop commodity servers.
    \item Instead of sending the data to the machine with the logic to execute, we send the logic to the data to execute, specifically, the server.
\end{itemize}

\subsubsection{How MapReduce works}

MapReduce is executed in the following order:

\begin{itemize}
    \item Split: Divides the dataset into multiple data batches.
    \item Map: Maps every element within each data batch to a $<key,value>$ pair.
    \item Await Completion: Wait for all data batches to finish mapping the pairs.
    \item Combine: Generates $<key,value>$ pairs in the form of a list (e.g., $[[A,1],[A,1]]$)
    \item Partition: Determines which reducer should handle each key.
    It uses a hash function (e.g., hash(key) \%num\_reducers) to distribute keys evenly.
    \item Reduce: Processes every data assigned to it and return the output.
\end{itemize}

\subsection{First subtask}
\label{subsec:first-subtask}

In the first subtask, we are assigned to store the data on Hadoop Distributes File System (HDFS).
After which we will implement a Hadoop MapReduce program in Java to discover groups of customers going shopping at the same date.

\subsubsection{The Mapper Class} CustomerGroupByDateMapper

The Mapper class is responsible for reading input data and emitting key-value pairs.
Key aspects of its implementation include:

\begin{itemize}
    \item Input Processing: The input data is a CSV file with seven columns including Member\_number (customer ID) and Date (transaction date).
    \item Filtering Headers: The Mapper ignores lines where Member\_number is a header.
    \item Emitting Key-Value Pairs: The transaction date is used as the key, and the customer ID is used as the value.
    This allows all customer transactions on a given date to be grouped together during the shuffle and sort phase.
\end{itemize}

Example Output from Mapper:
\begin{center}
(01/01/2014, 12345)\\
(01/01/2014, 67890)\\
(03/01/2014, 54321)
\end{center}

\subsubsection{The Reducer Class} CustomerGroupByDateReducer

The Reducer class is responsible for aggregating the values emitted by the Mapper for each unique key.
Key aspects of its implementation include:

\begin{itemize}
    \item Collecting Unique Customer IDs: The reducer stores customer IDs in a HashSet to ensure uniqueness.
    \item Joining Values: The unique customer IDs are converted into a comma-separated string.
    \item Emitting Results: The final output consists of the transaction date as the key and the list of unique customer IDs as the value.
\end{itemize}

Example Output from Reducer:
\begin{center}
(01/01/2014, 12345,67890)\\
(03/01/2014, 54321)
\end{center}

\subsubsection{Driver Program (Main Method)}

The driver program configures and executes the MapReduce job.
It performs the following tasks:

\begin{itemize}
    \item Setting up the Job: The job is named \("\)Customer Date Groups\("\) and configured to use GroupMapReduce as the main class.
    \item Setting Mapper and Reducer: The Mapper and Reducer classes are assigned appropriately
    \item Defining Input and Output: The input and output paths are provided as command-line arguments
    \item Job Execution: The job is submitted to Hadoop for execution, and the program exits based on its success or failure.
\end{itemize}

\subsection{Second subtask}
\label{subsec:second-subtask}

In the second subtask, we are assigned to implement the A-Priori algorithm to identify frequent customer pairs in the form of 02 Hadoop MapReduce programs, each corresponding to a pass.

\subsubsection{The First Pass} Identifying Frequent Individual Customers

\begin{itemize}
    \item Mapper Class: AprioriFirstPassMapper
    \begin{itemize}
        \item Function: Reads transaction data and emits each customer ID as a key with a value of 1.
        \item Filtering: Skips header lines and ensures valid data is processed.
        \item Example Output from Mapper:
        \begin{center}
        (12345, 1)\\
        (67890, 1)\\
        (12345, 1)
        \end{center}
    \end{itemize}

    \item Reducer Class: AprioriFirstPassReducer
    \begin{itemize}
        \item Function: Aggregates the occurrences of each customer ID.
        \item Filtering: Only customers meeting the support threshold (minimum occurrences) are retained.
        \item Example Output from Reducer:
        \begin{center}
        (12345, 2)\\
        (67890, 1)
        \end{center}
    \end{itemize}
\end{itemize}

\subsubsection{The Second Pass} Identifying Frequent Customer Pairs

\begin{itemize}
    \item Mapper Class: AprioriSecondPassMapper
    \begin{itemize}
        \item Setup: Loads frequent customers from the first pass output using Hadoop’s distributed cache.
        \item Processing: Reads transactions and filters out customers that did not meet the first pass threshold.
        \item Pair Generation: Creates all possible pairs of frequent customers.
        \item Example Output from Mapper:
        \begin{center}
        (12345, 67890, 1)\\
        (12345, 54321, 1)
        \end{center}
    \end{itemize}

    \item Reducer Class: AprioriSecondPassReducer
    \begin{itemize}
        \item Function: Aggregates occurrences of customer pairs and filters based on the support threshold.
        \item Example Output from Reducer:
        \begin{center}
        (12345, 67890, 3)
        \end{center}
    \end{itemize}
\end{itemize}

\subsubsection{Driver Program (Main Method)}

\begin{itemize}
    \item First Pass Execution:
    \begin{itemize}
        \item Runs the first MapReduce job to determine frequent individual customers.
        \item Saves the output for use in the second pass.
    \end{itemize}

    \item Second Pass Execution:
    \begin{itemize}
        \item Loads the first pass results as cached data.
        \item Runs the second MapReduce job to find frequent customer pairs.
    \end{itemize}
\end{itemize}

    \section{Second Task: PCY Algorithm for Frequent Items}
    \label{sec:second-task}
    \subsection{Overview of PCY}
\label{subsec:overview-of-pcy}
Frequent itemset mining is essential for discovering item associations in transactional data, such as market basket analysis.
The PCY algorithm improves efficiency by using hash buckets to reduce the computational cost of finding frequent item pairs.
This project applies the PCY algorithm to mine frequent itemsets and generate association rules based on support and confidence thresholds, using PySpark for scalable data processing.

The PCY algorithm is based on two key passes through the data.
In the first pass, frequent individual items are identified and counted.
In the second pass, frequent item pairs are counted, and hash buckets are used to prune less frequent pairs.
The hash function maps item pairs to buckets, and only pairs that have a sufficient bucket count are considered frequent.
This approach significantly reduces the number of pair comparisons and improves algorithm efficiency.

\begin{itemize}
    \item Step 1: Count individual items using the support threshold.
    \item Step 2: Count pairs of frequent items and hash them into buckets.
    \item Step 3: Prune item pairs that are not frequent based on the bucket counts.
    \item Step 4: Generate association rules using the confidence threshold.
\end{itemize}

\subsection{Implementation Details}
\label{subsec:implementation-details}

\subsubsection{Data Loading and Preprocessing}
Data is loaded using PySpark’s \texttt{read.csv} function.
Each transaction is represented as a basket, and the data is grouped by customer and date. The \texttt{collect\_set} function is used to create a list of items bought together in each transaction.

\subsubsection{First Pass: Counting Frequent Items}
In the first pass, each item’s frequency is counted, and only those items that meet the support threshold are considered frequent.
The item counts are stored in a dictionary, sorted in descending order.

\subsubsection{Second Pass: Counting Frequent Pairs}
During the second pass, the algorithm generates pairs from frequent items and counts their occurrences.
Hashing is applied to map pairs into buckets, and the bucket counts are used to prune pairs that do not meet the minimum support threshold.

\subsubsection{Association Rule Generation}
For each frequent pair, confidence is calculated as the ratio of the pair’s count to the individual item count.
Association rules are generated if the confidence meets the given threshold.
The rules are sorted by confidence.

\subsection{Experimental Results}
\label{subsec:experimental-results}
The PCY algorithm was executed on a transactional dataset of retail transactions, where each transaction (basket) represented a set of items purchased by a customer. The algorithm was applied with the following parameters:
\begin{itemize}
    \item Support threshold: 2 (minimum count for items to be considered frequent).
    \item Confidence threshold: 0.5 (minimum confidence for association rules).
\end{itemize}

\subsubsection{Frequent Items}
The first pass of the algorithm counted the occurrence of each item in the transactions. The following are the top 30 frequent items identified based on the support threshold:

\begin{table}[h]
    \centering
    \caption{Most Frequent Items with Their Counts}
    \setlength{\tabcolsep}{2pt} % Reduce column spacing
    \renewcommand{\arraystretch}{1} % Adjust row spacing
    \resizebox{120}{!}{ % Fit within column width
        \begin{tabular}{|l|c|}
            \hline
            \textbf{Item} & \textbf{Count} \\
            \hline
            Whole milk & 2363 \\
            Other vegetables & 1827 \\
            Rolls/buns & 1646 \\
            Soda & 1453 \\
            Yogurt & 1285 \\
            Root vegetables & 1041 \\
            Tropical fruit & 1014 \\
            Bottled water & 908 \\
            Sausage & 903 \\
            Citrus fruit & 795 \\
            Pastry & 774 \\
            Pip fruit & 734 \\
            \hline
        \end{tabular}
    }
    \label{tab:frequent_items}
\end{table}

\subsubsection{Frequent Item Pairs}
In the second pass, the algorithm counted pairs of frequent items across all transactions.
The top 30 frequent item pairs, sorted by frequency, are as follows:

\begin{table}[h]
    \centering
    \caption{Frequent Item Pairs with Their Counts}
    \setlength{\tabcolsep}{2pt} % Reduce column spacing
    \renewcommand{\arraystretch}{1} % Adjust row spacing
    \resizebox{160}{!}{ % Fit within column width
        \begin{tabular}{|l|c|}
            \hline
            \textbf{Pair} & \textbf{Count} \\
            \hline
            ('Whole milk', 'Other vegetables') & 222 \\
            ('Whole milk', 'Rolls/buns') & 209 \\
            ('Whole milk', 'Soda') & 174 \\
            ('Whole milk', 'Yogurt') & 167 \\
            ('Rolls/buns', 'Other vegetables') & 158 \\
            ('Soda', 'Other vegetables') & 145 \\
            ('Whole milk', 'Sausage') & 134 \\
            ('Whole milk', 'Tropical fruit') & 123 \\
            ('Yogurt', 'Other vegetables') & 121 \\
            ('Rolls/buns', 'Soda') & 121 \\
            ('Yogurt', 'Rolls/buns') & 117 \\
            ('Whole milk', 'Root vegetables') & 113 \\
            \hline
        \end{tabular}
    }
    \label{tab:frequent_pairs}
\end{table}

\subsubsection{Association Rules}
Once frequent item pairs were identified, association rules were generated based on the confidence threshold of 0.5. The confidence for each rule was computed by dividing the pair count by the count of the antecedent item. The top 30 association rules, sorted by confidence, are presented below:

\begin{table}[h]
    \centering
    \caption{Validated Association Rules}
    \setlength{\tabcolsep}{2pt} % Reduce column spacing
    \renewcommand{\arraystretch}{1} % Adjust row spacing
    \resizebox{180}{!}{ % Fit within column width
        \begin{tabular}{|l|c|}
            \hline
            \textbf{Rule} & \textbf{Confidence} \\
            \hline
            Preservation products → Soups & 1.00 \\
            Kitchen utensil → Pasta & 1.00 \\
            Kitchen utensil → Bottled water & 1.00 \\
            Kitchen utensil → Rolls/buns & 1.00 \\
            Bags → Yogurt & 0.50 \\
            \hline
        \end{tabular}
    }
    \label{tab:association-rules}
\end{table}

\subsubsection{Performance Evaluation}
The PCY algorithm effectively identified frequent items, pairs, and association rules with reasonable execution time and memory usage. Given the dataset size, the distributed nature of Spark ensured that the computation was scalable.
\begin{itemize}
    \item Time Complexity: The use of hashing significantly reduces the complexity of item pair generation, making the PCY algorithm faster than traditional algorithms such as the Apriori algorithm.
    \item Memory Usage: Memory usage was managed well by leveraging the distributed processing capabilities of Spark.
\end{itemize}

%\subsection{Contributions}
%The project was divided into the following tasks:
%\begin{itemize}
%    \item [Your Name]: Implemented the PCY algorithm, designed the class structure, and developed the data preprocessing functions.
%    \item [Group Member 2]: Focused on developing the association rule generation functionality and handled the confidence threshold calculation.
%    \item [Group Member 3]: Implemented data loading, Spark session management, and optimized the bucket counting logic.
%    \item [Lecturer Name]: Provided guidance on the overall project structure and reviewed the final implementation.
%\end{itemize}
%
%\subsection{Self-Evaluation}
%\begin{itemize}
%    \item Task Completion: The project was successfully implemented with full functionality. All required tasks were completed with a focus on modularity and scalability.
%    \item Estimated Score: Based on the project’s alignment with the requirements and completeness of the implementation, an estimated score of 95\% is assigned.
%\end{itemize}
%
%\subsection{Conclusion}
%This project successfully implemented the PCY algorithm for frequent itemset mining and association rule generation using PySpark. The algorithm efficiently identified frequent items and pairs, and generated association rules based on the given support and confidence thresholds. The use of hash buckets in the second pass helped prune non-frequent item pairs, improving performance. The implementation was carried out in an object-oriented manner to support scalability and maintainability. Further optimizations, such as parallelizing certain steps and optimizing memory usage, could enhance performance for larger datasets.
%
%\subsection{References}
%\begin{enumerate}
%    \item Park, J. S., & Chen, M. S. (1995).
%    Using a hash table to eliminate candidates in a frequent itemset mining algorithm.
%    IEEE Transactions on Knowledge and Data Engineering, 7(3), 464-472.
%    \item Han, J., Pei, J., & Yin, Y. (2000).
%    Mining frequent patterns without candidate generation.
%    ACM SIGMOD Record, 29(2), 1-12.
%    \item PySpark Documentation. (2025).
%    PySpark API Documentation.
%    Retrieved from https://spark.apache.org/docs/latest/api/python.
%\end{enumerate}

    \section{Third Task: MinHashLSH for Similar Dates}
    \label{sec:third-task}
    \subsection{Approach}
\label{subsec:approach}

Firstly, we will go through the theoretical basis and its possible implementation
in the context of our task, after which, we will see the algorithm in action.

\subsection{Theoretical basis}
\label{subsec:theoretical-basis}

The core of MinHashLSH algorithm is the utilization of two concepts: MinHash signatures and locality-sensitive hashing (LSH) to create and effective algorithm for detecting similar sets.
The first concept describes the process of hashing the dataset into a more manageable ``signature'' for each set.
While the latter describes how these ``signatures'' will be stored to achieve the expected result.

\subsection{Jaccard distance and Shingling}
\label{subsec:jaccard-distance-and-shingling}

Before performing any calculations to any data, we must convert the raw data into a distinguished vector before using any distance calculation between the two sets to check for their similarity.
Shingling performs this task by breaking down text data into smaller units to create ``shingles'' before hashing them into their representation in the form of a binary vector.

The Jaccard distance describes the similarity between two different sets and is represented by the following formula:
\begin{equation}
    \label{eq:equation}
    d_J(A, B) = \frac{|A \cup B| - |A \cap B|}{|A \cup B|} = 1 - J(A, B)
\end{equation}

With the result ranging from 0 to 1, we can determine whether the sets in question are related to each other or not for bucket assignment.

\subsection{MinHash Signature}
\label{subsec:minhash-signature}

To determine if the pairs are similar or not, we need a way to convert the raw data into usable data for the algorithm to process.
In this case,MinHashing is used to perform this task.
The binary vector, created through a process of shingling, is converted into a signature vector.
Note that if these sets 2 are similar, their signature vector will also have some similarities, and these can be utilized by the algorithm to sort these sets into their suitable buckets.

\subsection{Locality Sensitive Hashing}
\label{subsec:locality-sensitive-hashing}

The idea of LSH when dealing with the problem of finding similar pairs or sets is to maximize the probability of collision in a bucket due to that fact that the hash-code for these sets would be indifferent (if these sets are the similar) and therefore they should be in the same bucket.
We can achieve this by breaking the hash-code down even more into subsequences of hash-code that has a higher chance of being similar, giving us a higher chance of finding similar pairs, improving the effectiveness of the algorithm at solving the problem.

    \section*{Acknowledgment}

    The preferred spelling of the word ``acknowledgment'' in America is without an ``e'' after the ``g''.
    Avoid the stilted expression ``one of us (R. B. G.) thanks $\ldots$''.
    Instead, try ``R. B. G. thanks$\ldots$''.
    Put sponsor acknowledgments in the unnumbered footnote on the first page.

    \section*{References}

    Please number citations consecutively within brackets~\cite{b1}.
    The sentence punctuation follows the bracket~\cite{b2}.
    Refer simply to the reference number, as in~\cite{b3}---do not use ``Ref. \cite{b3}'' or ``reference~\cite{b3}'' except at the beginning of a sentence: ``Reference~\cite{b3} was the first $\ldots$''

    Number footnotes separately in superscripts.
    Place the actual footnote at
    the bottom of the column in which it was cited.
    Do not put footnotes in the abstract or reference list.
    Use letters for table footnotes.

    Unless there are six authors or more give all authors' names; do not use ``et al.''.
    Papers that have not been published, even if they have been submitted for publication, should be cited as ``unpublished''~\cite{b4}.
    Papers that have been accepted for publication should be cited as ``in press''~\cite{b5}.
    Capitalize only the first word in a paper title, except for proper nouns and element symbols.

    For papers published in translation journals, please give the English
    citation first, followed by the original foreign-language citation~\cite{b6}.

    \begin{thebibliography}{00}
        \bibitem{b1} G. Eason, B. Noble, and I. N. Sneddon, ``On certain integrals of Lipschitz-Hankel type involving products of Bessel functions,'' Phil. Trans. Roy. Soc. London, vol. A247, pp. 529--551, April 1955.
        \bibitem{b2} J. Clerk Maxwell, A Treatise on Electricity and Magnetism, 3rd ed., vol. 2. Oxford: Clarendon, 1892, pp.68--73.
        \bibitem{b3} I. S. Jacobs and C. P. Bean, ``Fine particles, thin films and exchange anisotropy,'' in Magnetism, vol. III, G. T. Rado and H. Suhl, Eds. New York: Academic, 1963, pp. 271--350.
        \bibitem{b4} K. Elissa, ``Title of paper if known,'' unpublished.
        \bibitem{b5} R. Nicole, ``Title of paper with only first word capitalized,'' J. Name Stand. Abbrev., in press.
        \bibitem{b6} Y. Yorozu, M. Hirano, K. Oka, and Y. Tagawa, ``Electron spectroscopy studies on magneto-optical media and plastic substrate interface,'' IEEE Transl. J. Magn. Japan, vol. 2, pp. 740--741, August 1987 [Digests 9th Annual Conf. Magnetics Japan, p. 301, 1982].
        \bibitem{b7} M. Young, The Technical Writer's Handbook. Mill Valley, CA: University Science, 1989.
    \end{thebibliography}
    \vspace{12pt}
    \color{red}
    IEEE conference templates contain guidance text for composing and formatting conference papers. Please ensure that all template text is removed from your conference paper prior to submission to the conference. Failure to remove the template text from your paper may result in your paper not being published.

\end{document}